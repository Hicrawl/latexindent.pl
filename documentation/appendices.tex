% arara: pdflatex: {shell: yes, files: [latexindent]}
\appendix
	\section{Required Perl modules}\label{sec:requiredmodules}
	 If you intend to use \texttt{latexindent.pl} and \emph{not} one of the
	 supplied standalone executable files, then you will need a few standard Perl modules --
	 if you can run the minimum code in \cref{lst:helloworld} (\texttt{perl helloworld.pl}) then you will be able to run
	 \texttt{latexindent.pl}, otherwise you may need to install the missing modules -- see
	 \cref{sec:module-installer,sec:manual-module-instal}.

	 \begin{cmhlistings}[style=tcblatex,language=Perl]{\texttt{helloworld.pl}}{lst:helloworld}
#!/usr/bin/perl

use strict;
use warnings;
use utf8;
use PerlIO::encoding;
use Unicode::GCString;
use open ':std', ':encoding(UTF-8)';
use Text::Wrap;
use Text::Tabs;
use FindBin;
use YAML::Tiny;
use File::Copy;
use File::Basename;
use File::HomeDir;
use Getopt::Long;
use Data::Dumper;
use List::Util qw(max);

print "hello world";
exit;
\end{cmhlistings}

	\subsection{Module installer script}\label{sec:module-installer}
		\announce{2018-01-13}{perl module helper script} \texttt{latexindent.pl} ships with a helper script that will
		install any missing \texttt{perl} modules on your system; if you run
		\begin{commandshell}
perl latexindent-module-installer.pl
\end{commandshell}
		or
		\begin{dosprompt}
perl latexindent-module-installer.pl
 \end{dosprompt}
		then, once you have answered \texttt{Y}, the appropriate modules will be
		installed onto your distribution.

	\subsection{Manually installed modules}\label{sec:manual-module-instal}
		Manually installing the modules given in \cref{lst:helloworld} will vary depending on
		your operating system and \texttt{Perl} distribution.

	\subsubsection{Linux}
		Linux users may be interested in exploring Perlbrew \cite{perlbrew}; an example
		installation would be:
		\begin{commandshell}
sudo apt-get install perlbrew
perlbrew init
perlbrew install perl-5.28.1
perlbrew switch perl-5.28.1
sudo apt-get install curl
curl -L http://cpanmin.us | perl - App::cpanminus
cpanm YAML::Tiny
cpanm File::HomeDir
cpanm Unicode::GCString
\end{commandshell}

		For other distributions, the Ubuntu/Debian approach may work as follows
		\begin{commandshell}
sudo apt install perl
sudo cpan -i App::cpanminus
sudo cpanm YAML::Tiny
sudo cpanm File::HomeDir
sudo cpanm Unicode::GCString
\end{commandshell}
		or else by running, for example,
		\begin{commandshell}
sudo perl -MCPAN -e'install "File::HomeDir"'
\end{commandshell}

		If you are using Alpine, some \texttt{Perl} modules are not build-compatible
		with Alpine, but replacements are available through \texttt{apk}. For example,
		you might use the commands given in \cref{lst:alpine-install}; thanks to
		\cite{jun-sheaf} for providing these details.

		\begin{cmhlistings}[style=tcblatex,language=Bash]{\texttt{alpine-install.sh}}{lst:alpine-install}
# Installing perl
apk --no-cache add miniperl perl-utils

# Installing incompatible latexindent perl dependencies via apk
apk --no-cache add \
    perl-log-dispatch \
    perl-namespace-autoclean \
    perl-specio \
    perl-unicode-linebreak

# Installing remaining latexindent perl dependencies via cpan
apk --no-cache add curl wget make
ls /usr/share/texmf-dist/scripts/latexindent
cd /usr/local/bin && \
    curl -L https://cpanmin.us/ -o cpanm && \
    chmod +x cpanm
cpanm -n App::cpanminus
cpanm -n File::HomeDir
cpanm -n Params::ValidationCompiler
cpanm -n YAML::Tiny
cpanm -n Unicode::GCString
\end{cmhlistings}

		Users of NixOS might like to see \href{https://github.com/cmhughes/latexindent.pl/issues/222}{https://github.com/cmhughes/latexindent.pl/issues/222} for tips.
	\subsubsection{Mac}
		Users of the Macintosh operating system might like to explore the following commands, for
		example:
		\begin{commandshell}
brew install perl
brew install cpanm

cpanm YAML::Tiny
cpanm File::HomeDir
cpanm Unicode::GCString
\end{commandshell}

	\subsubsection{Windows}
		Strawberry Perl users on Windows might use \texttt{CPAN client}. All of the modules are
		readily available on CPAN \cite{cpan}.

		\texttt{indent.log} will contain details of the location
		of the Perl modules on your system.  \texttt{latexindent.exe} is a standalone executable
		for Windows (and therefore does not require a Perl distribution) and caches copies of the
		Perl modules onto your system; if you wish to see where they are cached, use the
		\texttt{trace} option, e.g
		\begin{dosprompt}
latexindent.exe -t myfile.tex
 \end{dosprompt}

	\section{Updating the path variable}\label{sec:updating-path}
	 \texttt{latexindent.pl} has a few scripts (available at \cite{latexindent-home}) that can
	 update the \texttt{path} variables. Thank you to \cite{jasjuang} for this
	 feature. If you're on a Linux or Mac machine, then you'll want \texttt{CMakeLists.txt}
	 from \cite{latexindent-home}.
	\subsection{Add to path for Linux}
		To add \texttt{latexindent.pl} to the path for Linux, follow these steps:
		\begin{enumerate}
			\item download  \texttt{latexindent.pl} and its associated modules, \texttt{defaultSettings.yaml}, to
			      your chosen directory from \cite{latexindent-home} ;
			\item within your directory, create a directory called \texttt{path-helper-files} and download
			      \texttt{CMakeLists.txt} and \lstinline!cmake_uninstall.cmake.in! from
			      \cite{latexindent-home}/path-helper-files to this directory;
			\item run
			      \begin{commandshell}
ls /usr/local/bin
\end{commandshell}
			      to see what is \emph{currently} in there;
			\item run the following commands
			      \begin{commandshell}
sudo apt-get install cmake
sudo apt-get update && sudo apt-get install build-essential
mkdir build && cd build
cmake ../path-helper-files
sudo make install
\end{commandshell}
			\item run
			      \begin{commandshell}
ls /usr/local/bin
\end{commandshell}
			      again to check that \texttt{latexindent.pl}, its modules and \texttt{defaultSettings.yaml} have
			      been added.
		\end{enumerate}
		To \emph{remove} the files, run
		\begin{commandshell}
sudo make uninstall
\end{commandshell}
	\subsection{Add to path for Windows}
		To add \texttt{latexindent.exe} to the path for Windows, follow these steps:
		\begin{enumerate}
			\item download  \texttt{latexindent.exe}, \texttt{defaultSettings.yaml},  \texttt{add-to-path.bat} from
			      \cite{latexindent-home} to your chosen directory;
			\item open a command prompt and run the following command to see what is \emph{currently}
			      in your \lstinline!%path%! variable;
			      \begin{dosprompt}
echo %path%
          \end{dosprompt}
			\item right click on \texttt{add-to-path.bat} and \emph{Run as administrator};
			\item log out, and log back in;
			\item open a command prompt and run
			      \begin{dosprompt}
echo %path%
          \end{dosprompt}
			      to check that the appropriate directory has been added to your \lstinline!%path%!.
		\end{enumerate}
		To \emph{remove} the directory from your \lstinline!%path%!, run
		\texttt{remove-from-path.bat} as administrator.

	\section{logFilePreferences}\label{app:logfile-demo}
	 \Vref{lst:logFilePreferences} describes the options for customising the information given
	 to the log file, and we provide a few demonstrations here. Let's say that we start with
	 the code given in \cref{lst:simple}, and the settings specified in
	 \cref{lst:logfile-prefs1-yaml}.

	 \begin{minipage}{.35\linewidth}
		 \cmhlistingsfromfile{demonstrations/simple.tex}{\texttt{simple.tex}}{lst:simple}
	 \end{minipage}
	 \hfill
	 \begin{minipage}{.6\linewidth}
		 \cmhlistingsfromfile{demonstrations/logfile-prefs1.yaml}[yaml-TCB]{\texttt{logfile-prefs1.yaml}}{lst:logfile-prefs1-yaml}
	 \end{minipage}

	 If we run the following command (noting that \texttt{-t} is active)
	 \begin{commandshell}
latexindent.pl -t -l=logfile-prefs1.yaml simple.tex 
\end{commandshell}
	 then on inspection of \texttt{indent.log} we will find the snippet given in
	 \cref{lst:indentlog}.
	 \begin{cmhlistings}[style=tcblatex,morekeywords={TRACE}]{\texttt{indent.log}}{lst:indentlog}
       +++++
TRACE: environment found: myenv
       No ancestors found for myenv
       Storing settings for myenvenvironments
       indentRulesGlobal specified (0) for environments, ...
       Using defaultIndent for myenv
       Putting linebreak after replacementText for myenv
       looking for COMMANDS and key = {value}
TRACE: Searching for commands with optional and/or mandatory arguments AND key = {value}
       looking for SPECIAL begin/end
TRACE: Searching myenv for special begin/end (see specialBeginEnd)
TRACE: Searching myenv for optional and mandatory arguments
       ... no arguments found
       -----
     \end{cmhlistings}
	 Notice that the information given about \texttt{myenv} is `framed' using
	 \texttt{+++++} and \lstinline!-----! respectively.

	\section{Differences from Version 2.2 to 3.0}\label{app:differences}
	 There are a few (small) changes to the interface when comparing Version 2.2 to Version
	 3.0. Explicitly, in previous versions you might have run, for example,
	 \index{switches!-o demonstration}
	 \begin{commandshell}
latexindent.pl -o myfile.tex outputfile.tex
\end{commandshell}
	 whereas in Version 3.0 you would run any of the following, for example,
	 \index{switches!-o demonstration}
	 \begin{commandshell}
latexindent.pl -o=outputfile.tex myfile.tex
latexindent.pl -o outputfile.tex myfile.tex
latexindent.pl myfile.tex -o outputfile.tex 
latexindent.pl myfile.tex -o=outputfile.tex 
latexindent.pl myfile.tex -outputfile=outputfile.tex 
latexindent.pl myfile.tex -outputfile outputfile.tex 
\end{commandshell}
	 noting that the \emph{output} file is given \emph{next to} the
	 \texttt{-o} switch.

	 The fields given in \cref{lst:obsoleteYaml} are \emph{obsolete} from Version 3.0
	 onwards. \cmhlistingsfromfile{demonstrations/obsolete.yaml}[yaml-obsolete]{Obsolete YAML fields from Version 3.0}{lst:obsoleteYaml}

	 There is a slight difference when specifying indentation after headings; specifically, we
	 now write \texttt{indentAfterThisHeading} instead of \texttt{indent}. See
	 \cref{lst:indentAfterThisHeadingOld,lst:indentAfterThisHeadingNew}

	 \begin{minipage}{.45\textwidth}
		 \cmhlistingsfromfile{demonstrations/indentAfterThisHeadingOld.yaml}[yaml-TCB]{\texttt{indentAfterThisHeading} in Version 2.2}{lst:indentAfterThisHeadingOld}
	 \end{minipage}%
	 \hfill
	 \begin{minipage}{.45\textwidth}
		 \cmhlistingsfromfile{demonstrations/indentAfterThisHeadingNew.yaml}[yaml-TCB]{\texttt{indentAfterThisHeading} in Version 3.0}{lst:indentAfterThisHeadingNew}
	 \end{minipage}%

	 To specify \texttt{noAdditionalIndent} for display-math environments in Version 2.2, you
	 would write YAML as in \cref{lst:noAdditionalIndentOld}; as of Version 3.0, you would write YAML
	 as in \cref{lst:indentAfterThisHeadingNew1} or, if you're using \texttt{-m} switch,
	 \cref{lst:indentAfterThisHeadingNew2}.
	 \index{specialBeginEnd!update to displaymath V3.0}

	 \begin{minipage}{.45\textwidth}
		 \cmhlistingsfromfile{demonstrations/noAddtionalIndentOld.yaml}[yaml-TCB]{\texttt{noAdditionalIndent} in Version 2.2}{lst:noAdditionalIndentOld}
	 \end{minipage}%
	 \hfill
	 \begin{minipage}{.45\textwidth}
		 \cmhlistingsfromfile{demonstrations/noAddtionalIndentNew.yaml}[yaml-TCB]{\texttt{noAdditionalIndent} for \texttt{displayMath} in Version 3.0}{lst:indentAfterThisHeadingNew1}

		 \cmhlistingsfromfile{demonstrations/noAddtionalIndentNew1.yaml}[yaml-TCB]{\texttt{noAdditionalIndent} for \texttt{displayMath} in Version 3.0}{lst:indentAfterThisHeadingNew2}
	 \end{minipage}%

	 \mbox{}\hfill	  \begin{minipage}{.25\textwidth}
		 \hrule

		 \hfill\itshape End\\\mbox{}\hfill\mbox{}\rlap{\hfill\includegraphics{logo}}

	 \end{minipage}
